\documentclass[10pt,a4paper,ragged2e,withhyper]{altacv}

%% AltaCV uses the fontawesome5 and academicons fonts
%% and packages.
%% See http://texdoc.net/pkg/fontawesome5 and http://texdoc.net/pkg/academicons for full list of symbols. You MUST compile with XeLaTeX or LuaLaTeX if you want to use academicons.

%% Fork v1.6.5c: Overwriting sloppy environment to ignore any spaces and be used to solve overlapping cvtags
\newenvironment{sloppypar*}{\sloppy\ignorespaces}{\par}

% Change the page layout if you need to
\geometry{left=1.2cm,right=1.2cm,top=1cm,bottom=1cm,columnsep=0.75cm}

% The paracol package lets you typeset columns of text in parallel
\usepackage{paracol}

% Change the font if you want to, depending on whether
% you're using pdflatex or xelatex/lualatex
\ifxetexorluatex
  % If using xelatex or lualatex:
  \setmainfont{Roboto Slab}
  \setsansfont{Lato}
  \renewcommand{\familydefault}{\sfdefault}
\else
  % If using pdflatex:
  \usepackage[rm]{roboto}
  \usepackage[defaultsans]{lato}
  % \usepackage{sourcesanspro}
  \renewcommand{\familydefault}{\sfdefault}
\fi

% Fork (before v1.6.5a): Change the color codes to test your personal variant on any mode
\ifdarkmode%
  \definecolor{PrimaryColor}{HTML}{C69749}
  \definecolor{SecondaryColor}{HTML}{D49B54}
  \definecolor{ThirdColor}{HTML}{1877E8}
  \definecolor{BodyColor}{HTML}{ABABAB}
  \definecolor{EmphasisColor}{HTML}{ABABAB}
  \definecolor{BackgroundColor}{HTML}{191919}
\else%
  \definecolor{PrimaryColor}{HTML}{001F5A}
  \definecolor{SecondaryColor}{HTML}{0039AC}
  \definecolor{ThirdColor}{HTML}{F3890B}
  \definecolor{BodyColor}{HTML}{666666}
  \definecolor{EmphasisColor}{HTML}{2E2E2E}
  \definecolor{BackgroundColor}{HTML}{E2E2E2}
\fi%

\colorlet{name}{PrimaryColor}
\colorlet{tagline}{SecondaryColor}
\colorlet{heading}{PrimaryColor}
\colorlet{headingrule}{ThirdColor}
\colorlet{subheading}{SecondaryColor}
\colorlet{accent}{SecondaryColor}
\colorlet{emphasis}{EmphasisColor}
\colorlet{body}{BodyColor}
\pagecolor{BackgroundColor}

% Change some fonts, if necessary
\renewcommand{\namefont}{\Huge\rmfamily\bfseries}
\renewcommand{\personalinfofont}{\small\bfseries}
\renewcommand{\cvsectionfont}{\LARGE\rmfamily\bfseries}
\renewcommand{\cvsubsectionfont}{\large\bfseries}

% Change the bullets for itemize and rating marker
% for \cvskill if you want to
\renewcommand{\itemmarker}{{\small\textbullet}}
\renewcommand{\ratingmarker}{\faCircle}

%% sample.bib contains your publications
%% \addbibresource{main.bib}

\begin{document}
\name{Debjit Pal}
\tagline{B.Tech Graduate}
%% You can add multiple photos on the left or right
%% photo centreing is automatic
\photoL{4cm}{debjit-pal}

\personalinfo{
    \email{debjitpal5040@gmail.com}\smallskip
    \phone{+91-86977-85380}
    \location{Kolkata, India}\\
    \linkedin{linkedinUser}
    \github{githubUser}
    \npm{npmUser}
    \dev{devtoUser}
    \homepage{nicolasomar.me}
    \medium{nicolasomar}
    %% You MUST add the academicons option to \documentclass, then compile with LuaLaTeX or XeLaTeX, if you want to use \orcid or other academicons commands.
    % \orcid{0000-0000-0000-0000}
    %% You can add your own arbtrary detail with
    %% \printinfo{symbol}{detail}[optional hyperlink prefix]
    % \printinfo{\faPaw}{Hey ho!}[https://example.com/]
    %% Or you can declare your own field with
    %% \NewInfoFiled{fieldname}{symbol}[optional hyperlink prefix] and use it:
    % \NewInfoField{gitlab}{\faGitlab}[https://gitlab.com/]
    % \gitlab{your_id}
}

\makecvheader
%% Depending on your tastes, you may want to make fonts of itemize environments slightly smaller
% \AtBeginEnvironment{itemize}{\small}

%% Set the left/right column width ratio to 6:4.
\columnratio{0.25}

% Start a 2-column paracol. Both the left and right columns will automatically
% break across pages if things get too long.
\begin{paracol}{2}
    % ----- TECH STACK -----
    \cvsection{TECH STACK}

    \begin{sloppypar*}
        \cvtags{One, Two, Three, Four, Five, Six, Seven, Eight, Nine, Ten}
        \medskip

        \cvtags{Red, Yellow, Blue, Green, Violet, Orange}
    \end{sloppypar*}

    % ----- LEARNING -----
    \cvsection{Learning}
    \begin{sloppypar}
        \cvtags{Uno, Dos, Tres, Cuatro, Cinco}
        \medskip

        \cvtags{Rojo, Amarillo, Azul, Verde, Violeta}
    \end{sloppypar}

    % ----- LANGUAGES -----
    \cvsection{Languages}
    \cvlang{Bengali}{Native}\\
    \divider
    \cvlang{English}{Fluent}\\
    \divider
    \cvlang{Hindi}{Proficient}

    % ----- REFERENCES -----
    \cvsection{References}
    \cvref{Prof.\ Alpha Beta}{Institute}{a.beta@university.edu}
    \divider

    \cvref{Boss\ Gamma Delta}{Business}{g.delta@business.com}

    % ----- MOST PROUD -----
    % \cvsection{Most Proud of}

    % \cvachievement{\faTrophy}{Fantastic Achievement}{and some details about it}\\
    % \divider
    % \cvachievement{\faHeartbeat}{Another achievement}{more details about it of course}\\
    % \divider
    % \cvachievement{\faHeartbeat}{Another achievement}{more details about it of course}

    % \cvsection{A Day of My Life}

    % Adapted from @Jake's answer from http://tex.stackexchange.com/a/82729/226
    % \wheelchart{outer radius}{inner radius}{
    % comma-separated list of value/text width/color/detail}
    % \wheelchart{1.5cm}{0.5cm}{%
    %   6/8em/accent!30/{Sleep,\\beautiful sleep},
    %   3/8em/accent!40/Hopeful novelist by night,
    %   8/8em/accent!60/Daytime job,
    %   2/10em/accent/Sports and relaxation,
    %   5/6em/accent!20/Spending time with family
    % }

    % use ONLY \newpage if you want to force a page break for
    % ONLY the current column
    \newpage

    %% Switch to the right column. This will now automatically move to the second
    %% page if the content is too long.
    \switchcolumn

    % ----- ABOUT ME -----
    \cvsection{About Me}
    \begin{quote}
        Lorem ipsum dolor sit amet, consectetur adipiscing elit, sed do eiusmod tempor incididunt ut labore et dolore magna aliqua.
    \end{quote}

    % ----- EXPERIENCE -----
    \cvsection{Experience}
    \cvevent{Charge}{Company}{Mm YYYY -- Mm YYYY}{City, Country}
    \begin{itemize}
        \item First achievement
        \item Second achievement
        \item Third achievement
    \end{itemize}
    \divider

    \cvevent{Charge}{Company}{Mm YYYY -- Mm YYYY}{City, Country}
    \begin{itemize}
        \item First achievement
        \item Second achievement
        \item Third achievement
    \end{itemize}

    % ----- EDUCATION -----
    \cvsection{Education}
    \cvevent{B.tech in CSE}{Kalyani Government Engineering College}{Aug 2019 -- Jul 2023}{Kalyani, Nadia}
    \begin{itemize}
        \item CGPA: 9.31 / 10
    \end{itemize}

    \cvevent{Higher Secondary}{Uttarpara Government High School}{Aug 2017 -- May 2019}{Uttarpara, Hooghly}
    \begin{itemize}
        \item GPA: 88.86
    \end{itemize}

    \cvevent{Secondary}{Baranagore Ramakrishna Mission}{Jan 2011 -- May 2017}{Baranagore, North 24 Parganas}
    \begin{itemize}
        \item GPA: 94
    \end{itemize}

    % ----- PROJECTS -----
    \cvsection{Projects}
    \cvevent{Project 1 }{\cvreference{\faGithub}{https://github.com/user/repo}\cvreference{| \faGlobe}{https://project-demo.com/}}{Mm YYYY -- Mm YYYY}{}
    \begin{itemize}
        \item Item 1
        \item Item 2
    \end{itemize}
    \divider

    \cvevent{Project 2 }{\cvreference{\faGitlab}{https://gitlab.com/user/repo}\cvreference{| \faGlobe}{https://project-demo.com/}}{Mm YYYY -- Mm YYYY}{}
    \begin{itemize}
        \item Item 1
        \item Item 2
    \end{itemize}
\end{paracol}
\end{document}